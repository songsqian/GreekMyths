\documentclass{article}
\def\xcolorversion{2.00}
\def\xkeyvalversion{1.8}

\usepackage[version=0.96]{pgf}
\usepackage{tikz,url}
\usetikzlibrary{arrows,shapes,snakes,automata,backgrounds,petri}
\usetikzlibrary{positioning}
\usepackage[latin1]{inputenc}
\begin{document}
\raggedright

\tikzstyle{block} = [rectangle, draw, fill=gray!10,
%    text width=3.25em,
 text centered, rounded corners, minimum height=2em]
\tikzstyle{line} = [draw, very thick, color=black!50, -latex']
\tikzstyle{cloud} = [draw, ellipse,fill=gray!30, node distance=2.cm,
    minimum height=1.75em, minimum width=3.5em]
\tikzstyle{lb} = [text width=1.2cm]

Robert Graves' \textit{The Greek Myths} was in a style that is not
very appealing to me. In each chapter, Graves presenta description of
a myth, followed by some notes.  To make the reading more effective, I
tried to turn the description and the notes into diagrams, so that I
can picture the connections among all the difficult-to-pronounce
names. As a result, the reading became a prolonged project, which
helped me to learn the used of the \LaTeX\ \texttt{tikz} package.

\section{The Pelasgian Creation Myth} 

\begin{figure}[htb]
\begin{center}
  \resizebox{0.95\textwidth}{!}{
\begin{tikzpicture}[scale=2, node distance = 2.5cm, text width=1.25cm, 
on grid, auto]

    % Place nodes
   \node [block, text width=2cm] (a) {Eurynome};
\node [lb, right of=a] (an) {The creator};
   \node [cloud, align=center, below of=a](b){Universal Egg};
   \node [cloud, right of=a,yshift=-1cm, xshift=1cm] (c) {Ophion};
\node [lb, right of=c] (cn) {North wind}; 

    \node [cloud, align=left, below of=b, xshift=-6cm] (b1) {Sun (illumination)};
\node [lb, below of=b1, yshift=0.5cm] (b1n) {Theia\&\\ Hyperion\\ (Helius)};

    \node [cloud, right of=b1] (b2) {Moon (enchantment)};
\node [lb, below of=b2, yshift=0.25cm] (b2n) {Phoebe\&\\ Atlas\\ (Selene)};

    \node [cloud, right of=b2] (b3) {Mars (growth)};
\node [lb, below of=b3, yshift=0.5cm] (b3n) {Dione\&\\ Crius\\ (Ares)};

    \node [cloud, right of=b3] (b4) {Mercury\\ (wisdom)};
\node [lb, below of=b4, yshift=0.25cm] (b4n) {Metis\&\\  Coens\\ (Hermes/\\ Apollo)};

    \node [cloud, right of=b4] (b5) {Jupiter (law)};
\node [lb, below of=b5, yshift=0.5cm] (b5n) {Themis\&\\ Eurymedon\\ (Zeus)};

    \node [cloud, right of=b5] (b6) {Venus (love)};
\node [lb, below of=b6, yshift=0.5cm] (b6n) {Tethys\&\\ Oceanus\\ (Aphrodite)};

    \node [cloud, right of=b6] (b7) {Saturn (peace)};
\node [lb, below of=b7, yshift=.5cm] (b7n) {Rhea\&\\ Cronus\\ (Cronus)};

    \node [cloud, below of=b,align=center, yshift=-4.5cm] (d) {Earth};
    \node [lb, below of=d, align=center,yshift=1.5cm] (d1) {Pelasgus
      of\\ Arcadia};
\node [lb, right of=d1] (d1n) {First\\ man};
    % Draw edges
    \path [line] (a) -- (b);
    \path [line] (c) -- (b);
%    \path [line] (b) -- (b1);
\draw [->, very thick] (b) .. controls ([xshift=-2cm] b) and
([xshift=0.5cm, yshift=1cm] b1) .. (b1); 
    \path [line] (b) -- (b2);
    \path [line] (b) -- (b3);
    \path [line] (b) -- (b4);
    \path [line] (b) -- (b5);
    \path [line] (b) -- (b6);
%    \path [line] (b) -- (b7);
\draw [->, very thick] (b) .. controls ([xshift=2cm] b) and
([xshift=-0.5cm, yshift=1cm] b7) .. (b7);
%    \path [line] (b) -- (d);
\draw[->, very thick] (b) .. controls ([xshift=-5.cm, yshift=1cm] b)
and ([xshift=-5cm] d) .. (d);


\end{tikzpicture}

} \end{center}
\caption{Time line of the Pelasgian creation myth}
\label{fig:dagsEX2}
\end{figure}

\end{document}
