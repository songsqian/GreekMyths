\documentclass{article}
\def\xcolorversion{2.00}
\def\xkeyvalversion{1.8}

\usepackage[version=0.96]{pgf}
\usepackage{tikz,url}
\usetikzlibrary{arrows,shapes,snakes,automata,backgrounds,petri}
\usetikzlibrary{positioning}
\usepackage[latin1]{inputenc}
\begin{document}
\raggedright

\tikzstyle{block} = [rectangle, draw, fill=gray!10,
%    text width=3.25em,
 text centered, rounded corners, minimum height=2em]
\tikzstyle{line} = [draw, very thick, color=black!50, -latex']
\tikzstyle{cloud} = [draw, ellipse,fill=gray!30, node distance=2.cm,
    minimum height=1.75em, minimum width=3.5em]
\tikzstyle{lb} = [text width=1.2cm]

Robert Graves' \textit{The Greek Myths} was in a style that is not
very appealing to me. In each chapter, Graves presenta description of
a myth, followed by some notes.  To make the reading more effective, I
tried to turn the description and the notes into diagrams, so that I
can picture the connections among all the difficult-to-pronounce
names. As a result, the reading became a prolonged project, which
helped me to learn the used of the \LaTeX\ \texttt{tikz} package.

\section{The Pelasgian Creation Myth} 

\begin{figure}[htb]
\begin{center}
  \resizebox{0.95\textwidth}{!}{
\begin{tikzpicture}[scale=2, node distance = 2.5cm, text width=1.25cm, 
on grid, auto]

    % Place nodes
   \node [block, text width=2cm] (a) {Eurynome};
\node [lb, right of=a] (an) {The creator};
   \node [cloud, align=center, below of=a](b){Universal Egg};
   \node [cloud, right of=a,yshift=-1cm, xshift=1cm] (c) {Ophion};
\node [lb, right of=c] (cn) {North wind}; 

    \node [cloud, align=left, below of=b, xshift=-6cm] (b1) {Sun (illumination)};
\node [lb, below of=b1, yshift=0.5cm] (b1n) {Theia\&\\ Hyperion\\ (Helius)};

    \node [cloud, right of=b1] (b2) {Moon (enchantment)};
\node [lb, below of=b2, yshift=0.25cm] (b2n) {Phoebe\&\\ Atlas\\ (Selene)};

    \node [cloud, right of=b2] (b3) {Mars (growth)};
\node [lb, below of=b3, yshift=0.5cm] (b3n) {Dione\&\\ Crius\\ (Ares)};

    \node [cloud, right of=b3] (b4) {Mercury\\ (wisdom)};
\node [lb, below of=b4, yshift=0.25cm] (b4n) {Metis\&\\  Coens\\ (Hermes/\\ Apollo)};

    \node [cloud, right of=b4] (b5) {Jupiter (law)};
\node [lb, below of=b5, yshift=0.5cm] (b5n) {Themis\&\\ Eurymedon\\ (Zeus)};

    \node [cloud, right of=b5] (b6) {Venus (love)};
\node [lb, below of=b6, yshift=0.5cm] (b6n) {Tethys\&\\ Oceanus\\ (Aphrodite)};

    \node [cloud, right of=b6] (b7) {Saturn (peace)};
\node [lb, below of=b7, yshift=.5cm] (b7n) {Rhea\&\\ Cronus\\ (Cronus)};

    \node [cloud, below of=b,align=center, yshift=-4.5cm] (d) {Earth};
    \node [lb, below of=d, align=center,yshift=1.5cm] (d1) {Pelasgus
      of\\ Arcadia};
\node [lb, right of=d1] (d1n) {First\\ man};
    % Draw edges
    \path [line] (a) -- (b);
    \path [line] (c) -- (b);
%    \path [line] (b) -- (b1);
\draw [->, very thick] (b) .. controls ([xshift=-2cm] b) and
([xshift=0.5cm, yshift=1cm] b1) .. (b1); 
    \path [line] (b) -- (b2);
    \path [line] (b) -- (b3);
    \path [line] (b) -- (b4);
    \path [line] (b) -- (b5);
    \path [line] (b) -- (b6);
%    \path [line] (b) -- (b7);
\draw [->, very thick] (b) .. controls ([xshift=2cm] b) and
([xshift=-0.5cm, yshift=1cm] b7) .. (b7);
%    \path [line] (b) -- (d);
\draw[->, very thick] (b) .. controls ([xshift=-5.cm, yshift=1cm] b)
and ([xshift=-5cm] d) .. (d);


\end{tikzpicture}

} \end{center}
\caption{Time line of the Pelasgian creation myth}
\label{fig:dagsEX2}
\end{figure}

\clearpage
\section{The Homeric Creation Myth}

\begin{figure}[htb]
\begin{center}
\resizebox{0.95\textwidth}{!}{
\begin{tikzpicture}[scale=2, node distance = 2.5cm, text width=1.25cm, 
on grid, auto]

    % Place nodes
   \node [block, text width=3cm] (a) {Black-Winged\\ Night};
\node [lb, right of=a] (an) {A Goddess};
   \node [cloud, align=center, text width=3cm, below of=a](b){Silver Egg\\ (in Darkness)};
   \node [cloud, right of=a,yshift=-1cm, xshift=1cm] (c) {Wind};
\node [lb, right of=c] (cn) {Order\\ Justice}; 

    \node [cloud, align=left, below of=b, text width=1cm] (d) {Eros/ \\ Phanes};
    \node [cloud, below of=d, text width=1cm] (d1) {Earth};
    \node [cloud, right of=d1, text width=1cm] (d2) {Sun};
    \node [cloud, left of=d1, text width=1cm] (d3) {Moon};

    \node [lb, right of=d, xshift=1.75cm, text width=4cm] (dn) {double-sexed\\golden-winged\\four-headed};
    \node [lb, right of=dn, text width=3cm] (dnn) {Pha\"{e}theon\\ Protogenus\\ 4 seasons};
    \node [lb, left of=d] (dn2) {with Mother Rhea};
    % Draw edges
    \path [line] (a) -- (b);
    \path [line] (c) -- (b);
    \path [line] (b) -- (d);
    \path [line] (d) -- (d1);
    \path [line] (d) -- (d2);
    \path [line] (d) -- (d3);
\end{tikzpicture}

}
\end{center}
\caption{Homeric and Ophic creation myth}
\label{fig:dagsEX}
\end{figure}

\clearpage
\section{The Olympian Creation Myth}

\begin{figure}[htb]
\begin{center}
\resizebox{0.95\textwidth}{!}{
\begin{tikzpicture}[scale=2, node distance = 2.5cm, text width=1.25cm, 
on grid, auto]

    % Place nodes
    \node [block] (a) {Chaos};
    \node [lb, right of=a] (an) {Patriachal myth};
    \node [cloud, below of=a](b){Mother Earth};
    \node [cloud, right of=a, xshift=3cm, yshift=-2cm] (c) {Uranus};
    \node [lb, right of=c, text width=1.5cm] (cn) {Sky\\ 1st father};

    \node [cloud, below of=c, text width=0.8cm] (c1) {Flower};
    \node [cloud, left of=c1, xshift=-0.5cm, text width=1cm] (c2) {Grass};
    \node [cloud, right of=c1, xshift=0.5cm, text width=1cm] (c3) {Tree};
    \node [cloud, below of=c1, text width=1cm] (c4) {Rivers\\Lakes\\Seas};
    \node [cloud, below of=c2, xshift=0.7cm, text width=1cm] (c5) {Beasts};
    \node [cloud, below of=c3, xshift=-0.7cm, text width=1cm] (c6) {Birds};

    \node [cloud, below of=c, xshift=-6cm, text width=3cm] (d)
    {Hundred-handed\\giants}; 

    \node [cloud, below of=d, text width=1cm] (d1) {Gyges\\(earth)};
    \node [cloud, left of=d1, text width=1cm] (d2) {Briareus\\(strong)};
    \node [cloud, right of=d1, text width=1cm] (d3) {Cottus};

    \node [cloud, below of=d1, text width=1.5cm] (e) {One-eyed\\Cyclopes};
    \node [cloud, below of=e, text width=1cm] (e1) {Steropes};
    \node [cloud, left of=e1, text width=1cm] (e2) {Brontes};
    \node [cloud, right of=e1, text width=1cm] (e3) {Arges};

    \node [lb, right of=e, text width=5cm, xshift=1.25cm] (f)
    {Odysseus\\the physical world};
    % Draw edges
    \path [line] (a) -- (b);
    \path [line] (b) -- (c);
    \path [line] (c) -- (d);
    \path [line] (d) -- (d1);
    \path [line] (d) -- (d2);
    \path [line] (d) -- (d3);
    \path [line] (c) -- (c1);
    \path [line] (c) -- (c2);
    \path [line] (c) -- (c3);
%    \path [line] (c) -- (c4);
    \draw [->, very thick] (c) .. controls
    ([xshift=0.475cm,yshift=-0.75cm] c) and
    ([xshift=0.475cm,yshift=0.5cm] c4) .. (c4); 
    \path [line] (c) -- (c5);
%    \path [line] (c) -- (c6);
    \draw [->, very thick] (c) .. controls
    ([xshift=0.25cm,yshift=-0.25cm] c) and
    ([xshift=0.25cm,yshift=0.25cm] c6) .. (c6); 
    \path [line] (c) -- (c5);
%    \path [line] (c) -- (e);
    \draw [->, very thick] (c) .. controls
    ([xshift=-0.35cm,yshift=-0.5cm] c) and
    ([xshift=1.5cm,yshift=0.5cm] e) .. (e); 

    \path [line] (e) -- (e1);
    \path [line] (e) -- (e2);
    \path [line] (e) -- (e3);

\end{tikzpicture}

}
\end{center}
\caption{The Olympian creation myth}
\label{fig:dagsEX}
\end{figure}


%\begin{table}[h!]
%\centering
%\caption{An example data file}
%\begin{tabular}{c|ccc} \hline
% & Day 1 & Day 2 & Day 3\\ \hline
%Site 1 & 20.1 & 21.5 & 30\\
%Site 2 & 15.2 & 31.0& 12\\
%Site 3 & 20 & 25 & 19\\
%Site 4 & 11 & 14 & 21\\ \hline
%\end{tabular}
%\label{tab:exdata}
%\end{table}


%\begin{table}[t]
%\centering
%\caption{An example data frame}
%\begin{tabular}{ccc}\hline
% TP& Site & Day \\ \hline
%20.1 &Site 1 & Day 1\\
%21.5 &Site 1 & Day 2\\
%30   &Site 1 & Day 3\\
%15.2 &Site 2 & Day 1\\
%31   &Site 2 & Day 2\\
%12   &Site 2 & Day 3\\
%20   &Site 3 & Day 1\\
%25   &Site 3 & Day 2\\
%19   &Site 3 & Day 3\\
%11   &Site 4 & Day 1\\
%14   &Site 4 & Day 2\\
%21   &Site 4 & Day 3\\ \hline
%\end{tabular}
%\label{tab:exdataframe}
%\end{table}


\end{document}
